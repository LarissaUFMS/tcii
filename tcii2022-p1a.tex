\documentclass[12pt]{article}

\usepackage{amsmath,amsfonts,bezier,amstext,amsthm,amssymb}
\usepackage[utf8]{inputenc}
\usepackage[brazil]{babel}
\usepackage{geometry}
\usepackage{setspace}
\usepackage{xcolor}
\usepackage{comment}
\usepackage{bold-extra}

\geometry{top=15mm,bottom=17mm,left=30mm,right=25mm}
\setlength{\parskip}{1ex}
\setlength{\parindent}{1.2cm}

\definecolor{myblue}{HTML}{0066cc}
\definecolor{mygreen}{HTML}{009926}
\definecolor{myorange}{HTML}{ff8000}

\newcommand{\kw}[1]{{\color{myblue}\texttt{\textbf{#1}}}}
\newcommand{\sym}[1]{{\color{mygreen}\texttt{#1}}}
\newcommand{\hfile}[1]{{\color{myorange}\texttt{#1.h}}}
\newcommand{\cfile}[1]{{\color{myorange}\texttt{#1.cpp}}}
\newcommand{\R}[1]{\ensuremath{\mathbb{R}^{#1}}}
\newcommand{\E}[1]{\ensuremath{\mathbb{E}^{#1}}}
\newcommand{\point}[1]{\ensuremath{\mathcal{#1}}}
\newcommand{\pv}[1]{\ensuremath{\overrightarrow{\point{#1}}}}

\begin{document}

\pagestyle{empty}

\begin{center}
  Universidade Federal de Mato Grosso do Sul \\
  Faculdade de Computação \\[1em]
  {\bf\large TÓPICOS EM COMPUTAÇÃO II 2022} \\
  Paulo A. Pagliosa \\[1em]
  {\bf PROVA 1 - PARTE 1} \\
  19/05/2022
\end{center}

\vspace{5mm}
\noindent
Responda as questões abaixo em um arquivo texto com extensão \texttt{.txt}, identificado com seu nome, e o submeta via AVA.


\vspace{5mm}
\noindent
{\bf QUESTÃO 1 (1.0)}

\noindent
Qual é o tipo da variável \texttt{x} na declaração a seguir?\\[1.5ex]
\texttt{\kw{int}* \kw{const}\ (*x[3])(\kw{int const}* (*)(\kw{int}*[2]));}

\vspace{5mm}
\noindent
{\bf QUESTÃO 2 (1.0)}

\noindent
Explique o significado da declaração na linha 20 do arquivo \hfile{Shape}.

\vspace{5mm}
\noindent
{\bf QUESTÃO 3 (0.5)}

\noindent
Explique o significado da declaração na linha 24 do arquivo \hfile{Circle}.

\vspace{5mm}
\noindent
{\bf QUESTÃO 4 (0.5)}

\noindent
Explique como funciona o \kw{for}\ da linha 24 de \cfile{ShapeTest}.

\vspace{4mm}
\noindent
{\bf QUESTÃO 5 (1.0)}

\noindent
Identifique quais as mensagens enviadas na expressão da linha 25 de \cfile{ShapeTest} e, para cada uma, justifique qual o tipo de acoplamento (estático ou dinâmico).

\vspace{5mm}
\noindent
{\bf QUESTÃO 6 (1.0)}

\noindent
Considere as seguintes declarações:\\[1.5ex]
\texttt{\kw{constexpr auto}\ a = 5;}\\
\texttt{\kw{const auto}\ b = 5;}\\[1.5ex]
Seja a função na linha 43 do arquivo \hfile{FunctionTemplate}. Os resultados das chamadas da função com os argumentos \texttt{a} e \texttt{b} são avaliados em tempo de compilação ou execução? Justifique.

\vspace{5mm}
\noindent
{\bf QUESTÃO 7 (1.0)}

\noindent
Explique como funciona a semântica de cópia de ponteiros espertos em \hfile{SharedObject}.

\vspace{1cm}
\centerline{\bf Boa prova!}

\end{document}
