\documentclass[12pt]{article}

\usepackage{amsmath,amsfonts,bezier,amstext,amsthm,amssymb}
\usepackage[utf8]{inputenc}
\usepackage[brazil]{babel}
\usepackage{geometry}
\usepackage{setspace}
\usepackage{xcolor}
\usepackage{comment}
\usepackage{bold-extra}

\geometry{top=15mm,bottom=17mm,left=30mm,right=25mm}
\setlength{\parskip}{1ex}
\setlength{\parindent}{1.2cm}

\definecolor{myblue}{HTML}{0066cc}
\definecolor{mygreen}{HTML}{009926}
\definecolor{myorange}{HTML}{ff8000}

\newcommand{\kw}[1]{{\color{myblue}\texttt{\textbf{#1}}}}
\newcommand{\sym}[1]{{\color{mygreen}\texttt{#1}}}
\newcommand{\hfile}[1]{{\color{myorange}\texttt{#1.h}}}
\newcommand{\cfile}[1]{{\color{myorange}\texttt{#1.cpp}}}
\newcommand{\R}[1]{\ensuremath{\mathbb{R}^{#1}}}
\newcommand{\E}[1]{\ensuremath{\mathbb{E}^{#1}}}
\newcommand{\point}[1]{\ensuremath{\mathcal{#1}}}
\newcommand{\pv}[1]{\ensuremath{\overrightarrow{\point{#1}}}}

\begin{document}

\pagestyle{empty}

\begin{center}
  Universidade Federal de Mato Grosso do Sul \\
  Faculdade de Computação \\[1em]
  {\bf\large TÓPICOS EM COMPUTAÇÃO II 2022} \\
  Paulo A. Pagliosa \\[1em]
  {\bf PROVA 1 - PARTE 2} \\
  19/05/2022
\end{center}

\vspace{5mm}
\noindent
{\bf QUESTÃO ÚNICA (4.0)}

\noindent
Tomando a classe \sym{Vector} como referência, escreva um \kw{template} de classe (vamos chamar de \emph{classe paramétrica}) de vetores de elementos de um tipo \sym{T}:\\[1.5ex]
\kw{template}\ \texttt{<}\kw{typename}\ \sym{T}\texttt{>} \kw{class}\ \sym{Vector};\\[1.5ex]
Assuma que \sym{T}\ é tipo numérico (\kw{int}, \kw{float}\ ou \texttt{std::}\sym{complex}\texttt{<}\kw{float}\texttt{>}, por exemplo). Os membros da classe paramétrica serão os mesmos da classe \sym{Vector}\ desenvolvida em sala. A diferença é que, com a classe paramétrica, poderemos ter vetores de elementos numéricos de outros tipos além de \kw{float}. Contudo, diferente da classe \sym{Vector}, a cópia de objetos da classe paramétrica deverá obedecer a propriedade de \emph{independência}: após a operação de cópia \sym{v}\texttt{=}\sym{u}, qualquer alteração em \sym{u}\ não deve alterar o estado de \sym{v} e vice-versa. Não é isso que acontece com o operador de cópia implementada na classe \sym{Vector}, conforme se observa na execução da função de teste \texttt{vectorTest()} em \cfile{MatrixTest}: após a cópia na linha 21, \sym{u}\ é alterado na linha 24, e assim \sym{v}, como mostra a iteração da linha 27.

Para garantir a independência na cópia de vetores, vamos considerar a noção de \emph{copy-on-right}: \textbf{se} uma operação qualquer sobre um vetor, \sym{u}, alterar o estado do corpo do vetor, \sym{u}\texttt{.}\sym{\_body}, \textbf{e se} o contador de referência do corpo do vetor for maior que $1$ (isto é, \sym{u}\texttt{.}\sym{\_body} é ``usado'' por mais de um vetor), então efetua-se uma cópia profunda de \sym{u}\texttt{.}\sym{\_body}, cópia que será o novo corpo de \sym{u}. É sobre esse novo corpo que a operação será realizada.

Implemente a classe paramétrica em \hfile{Vector}, no lugar da classe \sym{Vector}. Os métodos não \kw{inline}\ da (agora também) classe paramétrica de corpo de vetor também ficam nesse arquivo e começam como em\\[1.5ex]
\kw{template}\ \texttt{<}\kw{typename}\ \sym{T}\texttt{>}\\
\sym{VectorBody}\texttt{<}\sym{T}\texttt{>*}\\
\sym{VectorBody}\texttt{<}\sym{T}\texttt{>::add(}$\ldots$\\[1.5ex]
Em um arquivo chamado \cfile{VectorTest}, escreva uma função de teste \texttt{vectorTest()}. a qual deve testar \textbf{todas} as funcionalidades da classe paramétrica para os tipos \kw{int}\ e \kw{float}. No mesmo arquivo, escreva uma função principal que chama a função de teste. Os arquivos \hfile{Vector}\ e \cfile{VectorTest} devem iniciar com comentários contendo o seu nome. Ao finalizar a implementação, submeta os dois arquivos via AVA.

\vspace{1cm}
\centerline{\bf Boa prova!}

\end{document}
