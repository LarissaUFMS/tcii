\documentclass[12pt]{article}

\usepackage{amsmath,amsfonts,bezier,amstext,amsthm,amssymb}
\usepackage[utf8]{inputenc}
\usepackage[brazil]{babel}
\usepackage{geometry}
\usepackage{setspace}
\usepackage{xcolor}
\usepackage{comment}
\usepackage{bold-extra}
\usepackage{array}

\geometry{top=15mm,bottom=17mm,left=30mm,right=25mm}
\setlength{\parskip}{1ex}
\setlength{\parindent}{1.2cm}

\definecolor{myblue}{HTML}{0066cc}
\definecolor{mygreen}{HTML}{009926}
\definecolor{myorange}{HTML}{ff8000}

\newcommand{\kw}[1]{{\color{myblue}\texttt{\textbf{#1}}}}
\newcommand{\sym}[1]{{\color{mygreen}\texttt{#1}}}
\newcommand{\hfile}[1]{{\color{myorange}\texttt{#1.h}}}
\newcommand{\cfile}[1]{{\color{myorange}\texttt{#1.cpp}}}
\newcommand{\R}[1]{\ensuremath{\mathbb{R}^{#1}}}
\newcommand{\E}[1]{\ensuremath{\mathbb{E}^{#1}}}
\newcommand{\point}[1]{\ensuremath{\mathcal{#1}}}
\newcommand{\pv}[1]{\ensuremath{\overrightarrow{\point{#1}}}}

\begin{document}

\pagestyle{empty}

\begin{center}
  Universidade Federal de Mato Grosso do Sul \\
  Faculdade de Computação \\[1em]
  {\bf\large TÓPICOS EM COMPUTAÇÃO II 2022} \\
  Paulo A. Pagliosa \\[1em]
  {\bf Prova Substitutiva} \\
  07/07/2022
\end{center}

\vspace{5mm}
\noindent
Responda as questões abaixo em um arquivo texto com extensão \texttt{.txt}, identificado com seu nome. Codifique a questão de implementação em um arquivo chamado \hfile{myprintf}, contendo comentários com seu nome. Escreva o restante de código em um arquivo com extensão \texttt{.cpp}, identificado com seu nome. Compacte todos os arquivos em um arquivo \texttt{.zip}, identificado com seu nome, e o submeta via AVA.

\vspace{5mm}
\noindent
{\bf QUESTÃO DE PROGRAMAÇÃO (3.5)}

\noindent
Escreva código para que o \kw{template}\ de função\\[1.5ex]
\texttt{\kw{template}\ <\kw{typename}... \sym{Args}>}\\
\texttt{\kw{void}\ myprintf(\kw{const char}* format, \sym{Args}\&\&... args);}\\[1.5ex]
instancie uma função \texttt{myprintf} para escrita na saída padrão que seja``segura''. O parâmetro \texttt{format} é uma cadeia de caracteres (exceto \texttt{\%}) que são copiados na saída padrão (por simplicidade,  não são considerados caracteres de controle tais como \texttt{\textbackslash{n}}). Quando um \texttt{\%} é encontrado, este deve ser seguido por um caractere \emph{especificador de conversão}. Se não houver caractere após \texttt{\%}, então a cadeia de formato está mal formada e uma exceção deve ser lançada. Se o especificador de conversão for \texttt{\%}, então \texttt{\%} é copiado para a saída padrão. Para qualquer outro especificador de conversão espera-se um argumento em \texttt{args} de um tipo válido para o especificador. Se não houver um argumento para um especificador de conversão, ou se o argumento for de um tipo inválido, ou se o especificador de conversão for inválido, uma exceção deve ser lançada. Os especificadores de conversão válidos são:
\begin{center}
\begin{tabular}{|c|m{0.3\textwidth}|}
\hline
Caractere & \multicolumn{1}{c|}{Tipos esperados}\\
\hline
\texttt{c} & \kw{char}\\
\hline
\texttt{s} & \kw{const char*}\newline\kw{const}\texttt{ std::\sym{string}\&}\\
\hline
\texttt{d}\ & \kw{int}\newline\kw{long}\newline\kw{short}\\
\hline
\texttt{u}\ & \kw{unsigned int}\newline\kw{unsigned long}\newline\kw{unsigned short}\\
\hline
\texttt{f}\ & \kw{float}\newline\kw{double}\\
\hline
\texttt{b}\ & \kw{bool}\\
\hline
\end{tabular}
\end{center}
Ao final, \texttt{\textbackslash{n}} é copiado na saída padrão. Escreva uma função de testes para seu \kw{template} (com tratamento de exceções).

\vspace{5mm}
\noindent
{\bf QUESTÃO 1 (2.5)}

\noindent
Descreva quais são os (seis) métodos definidos implicitamente pelo compilador para uma classe \sym{X}, quando são gerados e qual a funcionalidade \kw{default}\ de cada um.

\vspace{5mm}
\noindent
{\bf QUESTÃO 2 (1.0)}

\noindent
Quais as funcionalidades e diferenças de \texttt{std::move} e \texttt{std::forward}?

\vspace{5mm}
\noindent
{\bf QUESTÃO 3 (2.0)}

\noindent
Considere o seguinte código:\\[1.5ex]
\texttt{\kw{template}\ <\kw{typename}... \sym{Args}>}\\
\texttt{\kw{void}\ log(std::\sym{ostream}\& os, \sym{Args}\&\&... args)}\\
\texttt{\{}\\
\hspace*{1em}\texttt{(os << ... << std::forward<\sym{Args}>(args)) << '\textbackslash{n}';}\\
\texttt{\}}\\[1.5ex]
\texttt{\kw{using}\ \sym{strings} = std::\sym{vector}<std::\sym{string}>;}\\
\texttt{\kw{using}\ \sym{logfunc} = std::\sym{function}<\kw{void}(\kw{const}\ \sym{strings}\&)>;}\\[1.5ex]
\texttt{\kw{inline void}\ logText(\sym{logfunc}\ f) \{ f(\{"UFMS", "FACOM", "TCCII-2022"\}); \}}\\[1.5ex]
\texttt{\kw{inline void}\ testLog(std::\sym{ostream}\& os = std::cout)}\\
\texttt{\{}\\
\hspace*{1em}\texttt{logText([\&os](\kw{const}\ \sym{strings}\& s)}\\
\hspace*{2em}\texttt{\{}\\
\hspace*{3em}\texttt{\kw{for}\ (\sym{size\_t} n = s.size(), i = 0; i < n; ++i)}\\
\hspace*{4em}\texttt{log(os, i + 1, ':', s[i]);}\\
\hspace*{2em}\texttt{\});}\\
\texttt{\}}\\[1.5ex]
Reescreva a função \texttt{testLog} usando um \emph{functor} no lugar da expressão lambda.

\vspace{5mm}
\noindent
{\bf QUESTÃO 4 (1.0)}

\noindent
Explique a diferença entre \kw{struct}s e \kw{union}s.

\vspace{1cm}
\centerline{\bf Boa prova!}

\end{document}
